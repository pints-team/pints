\documentclass[12pt]{article}
\usepackage{amsmath, amssymb, bm}
\usepackage{timestamp}

\newcommand{\cvec}{\textbf{c}}
\newcommand{\evec}{\textbf{e}}
\newcommand{\fvec}{\textbf{f}}
\newcommand{\gvec}{\textbf{g}}
\newcommand{\honevec}{\textbf{h}_\textbf{1}}
\newcommand{\htwovec}{\textbf{h}_\textbf{2}}
\newcommand{\hvec}{\textbf{h}}
\newcommand{\lvec}{\textbf{l}}
\newcommand{\pvec}{\textbf{p}}
\newcommand{\qvec}{\textbf{q}}
\newcommand{\rvec}{\textbf{r}}
\newcommand{\svec}{\textbf{s}}
\newcommand{\vvec}{\textbf{v}}
\newcommand{\wvec}{\textbf{w}}
\newcommand{\xvec}{\textbf{x}}
\newcommand{\xnotvec}{\textbf{x}_\textbf{0}}
\newcommand{\xonevec}{\textbf{x}_\textbf{1}}
\newcommand{\xtwovec}{\textbf{x}_\textbf{2}}
\newcommand{\yvec}{\textbf{y}}
\newcommand{\zvec}{\textbf{z}}
\newcommand{\zerovec}{\textbf{0}}
\newcommand{\onevec}{\textbf{1}}
\newcommand{\gammavec}{\textbf{\gamma}}
\newcommand{\phivec}{\textbf{\phi}}
\newcommand{\psivec}{\textbf{\psi}}
\newcommand{\rhovec}{\textbf{\rho}}
\newcommand{\thetavec}{\textbf{\theta}}
\newcommand{\sensai}{\textsc{sensai}}
%
\newcommand{\Amat}{\textbf{A}}
\newcommand{\Bmat}{\textbf{B}}
\newcommand{\Cmat}{\textbf{C}}
\newcommand{\Fmat}{\textbf{F}}
\newcommand{\Imat}{\textbf{I}}
\newcommand{\Smat}{\textbf{S}}
\newcommand{\Tmat}{\textbf{T}}
\newcommand{\Vmat}{\textbf{V}}
\newcommand{\Wmat}{\textbf{W}}
\newcommand{\Xmat}{\textbf{X}}
\newcommand{\zeromat}{\textbf{0}}
\newcommand{\bzero}{\textbf{0}}
%
\newcommand{\reals}{\mathbb R}
\newcommand{\DD}{\displaystyle}
\newcommand{\norm}[1]{\left\lVert#1 \right\rVert_{2}}

%\newcommand{\xdim}{{\rm xdim}}
%\newcommand{\kdim}{{\rm kdim}}
\newcommand{\xdim}{{\rm M}}
\newcommand{\kdim}{{\rm K}}
\newcommand{\xonedim}{{\rm M_1}}
\newcommand{\xtwodim}{{\rm M_2}}


\begin{document}
\title{SENSAI: R0\_README file}
\author{Simon Tavener \& Michael Mikucki}
\maketitle
\hfill Date \& time: \timestamp
%\date{}

\pagestyle{empty}

\section{How to Define $R_0$ in {\sc sensai}}

{\sc sensai} is capable of automatically defining the basic reproduction ratio, $R_0$, as defined by the Next Generation method, for appropriate epidemiological models.  However, {\sc sensai} is not limited to infection modeling, so specific syntax is required so that {\sc sensai} recognizes if a model is compatible to the definition of $R_0$.  The following guide will instruct the user on how to edit the MuPAD templates so that {\sc sensai} will produce $R_0$ and its sensitivity analysis.

\begin{enumerate}
\item Edit the MuPAD templates to define your model equations.  These should be stored as the vector $g[i]$, the right-hand side of the equation for the variable $x[i]$.

\item Define which equations from $g$ define the dynamics of infected classes.  Store these indices in the variable $NextGen$.  Note that $NextGen$ must be written in \textit{matrix} syntax.  Consider the following examples.
 \begin{enumerate}
  \item For example, if the model includes three states, $S$, $I$, and $R$, in that order, $NextGen := matrix([2]);$.
  \item If the model has more than one equation describing an infected class, list them in the order they appear.  For example, if the model describes $S_1, I_1, R_1, S_2, I_2, R_2$ in that order, $NextGen := matrix([2,5]);$.
  \item \textit{If you do not wish to calculate $R_0$ for the model, define $NextGen = matrix([0])$, or let the first entry of $NextGen$ be 0.}
 \end{enumerate}

\item If the model has four or more infected classes, you may want to consider computing $R_0$ without its sensitivities.  $R_0$ will be a very lengthy expression for such models, and the derivatives will require a lot of time to compute.  If this is the case, define ``R0\_only" to be 1.  If you wish to calculate the sensitivities anyway, define ``R0\_only" = 0.
 \begin{enumerate}
  \item While running the {\sc sensai} GUI, you may encounter large delays in ``Create M{\sc atlab} files using MuPAD" if R0\_only = 0.  If your patience has run thin, you must terminate the program through the task manager.  The emergency stop in M{\sc{atlab}} of CTRL+c in the command window will not work, as the computation of $R_0$ is done externally in a MuPAD procedure call.
 \end{enumerate}

\item If the analytical expression for $R_0$ is already known, it may be faster (and more accurate if MuPAD can not solve $R_0$) to use this expression of $R_0$ for the quantity of interest ($qoi$) instead of re-deriving the expression during the ``Create M{\sc atlab} files using MuPAD" phase.

\end{enumerate}

\subsection{Possible Problems with $R_0$}

There are some examples in which the Next Generation construction of $R_0$ is not valid, or is not compatible with {\sc sensai}.  The following are possible problems the user might encounter when trying to define $R_0$.

\begin{enumerate}

\item \textbf{ Problems with ODE models.}  For $R_0$ to be valid, the model must satisfy the conditions of Theorem 2 in \textit{Reproduction numbers and sub-threshold endemic equilibria for compartmental models of disease transmission}, van den Driessche 2002.  Recall for ODEs, the Next Generation definition of $R_0 = \rho(FV^{-1})$ where $F = \DD\frac{\partial \mathcal{F}_i}{\partial x_j}(x^{\star}) \quad 1 \leq i,j \leq m$ describes new infections and $V = \DD\frac{\partial \mathcal{V}_i}{\partial x_j}(x^{\star}) \quad 1 \leq i,j \leq m$ describes transfer of existing infections, $x^{\star}$ is the disease-free equilibrium, the infected classes are $1,\dots,m$, and $\rho(\cdot)$ denotes the spectral radius operator.

 \begin{enumerate}

  \item The fecundity matrix $F$ is not nonnegative.  This is part of assumption (A1) of van den Driessche 2002.
  \item The transition matrix $V$ is singular.  This can occur if an equation is in the model as a placeholder, but the right-hand side is identically 0.  This state must be removed from the system for $R_0$ to be valid.
  \item The disease-free subspace is not invariant.  That is, infection can enter a disease-free population through a nonzero component in a state that is identified as disease-free.  This can occur in models with background infection rates, or in models where the infective classes are not identified properly.  This is assumption (A4).
  \item The equilibrium is not asymptotically stable in the absence of disease.  That is, if $\mathcal{F} = 0$, there is an eigenvalue of the Jacobian of the full system evaluated at $x^{\star}$ that has a positive real part.  This is assumption (A5).

 \end{enumerate}

\item \textbf{Problems with map models.}  For $R_0$ to be valid, the model must satisfy the conditions of Theorem 2.1 in \textit{The Basic Reproduction Number in Some Discrete-Time Epidemic Models}, Allen 2008.  Recall for maps, the Next Generation definition of $R_0 = \rho(F(I-T)^{-1})$, where $I$ is the $m\times m$ identity and $F$ and $-T$ are defined the same as $F$ and $V$ for ODEs, respectively.

 \begin{enumerate}

  \item The fecundity matrix $F$ is not nonnegative.
  \item The transition matrix $T$ is not nonnegative.
  \item The transition matrix $T$ is singular.  This can occur if an equation is in the model as a placeholder, but the right-hand side is identically 0.  This state must be removed from the system for $R_0$ to be valid.
  \item The transition matrix $T$ is not asymptotically stable.  That is, $\rho(T) \geq 1$.
  \item The equilibrium is not asymptotically stable in the absence of disease.  That is, $\rho(C) \geq 1$ where $C$ is the Jacobian of the right-hand side of the noninfectious states.

 \end{enumerate}

\end{enumerate}

Notice that assumptions (A2) and (A3) for ODE models are not automatically checked by {\sensai}.  These assumptions must be verified by the user, but are usually true.  For map models, the assumption of a unique DFE is not checked by {\sensai}, nor is the condition that $F+T$ is irreducible.  These should also be checked by the user to ensure a valid $R_0$.  It is difficult to check both of these conditions, but again, for most models, $F+T$ is irreducible based on the structure of $T$ having a nonzero main diagonal and a sub-diagonal and the structure of $F$ having a nonzero top row.

There are a number of reasons for any of the problems in lists 1 and 2 to occur.  Perhaps the model does not have a valid Next Generation construction of $R_0$.  If this is the case, some alternative means to calculate $R_0$ should be sought, if desired.  Alternatively, {\sc sensai} may not be able to recognize which terms describe new infections and belong in $\mathcal{F}$ and which terms describe transfer of existing infections and belong in $\mathcal{V}$ or $\mathcal{T}$.  The following criteria are used by {\sc sensai} to determine the placement of each term.  If the terms of the model will not be placed in the biologically correct vectors, {\sensai} fails to compute the Next Generation $R_0$.

\begin{enumerate}

\item If the term $X$ in an equation describing an infective class involves a state variable from a noninfectious class, $X \in \mathcal{F}$, unless the occurrence of the noninfectious state variable is part of a sum of all state variables (that is, the term is scaled by the total population).\\

\textbf{Important Note for MuPAD}:  At this point, MuPAD is not as effective as {\sf Maple} at analytical computations.  There may be examples (like the Hantavirus model) where MuPAD can not account for the scaling by the total population.  This problem will hopefully be fixed in a future version of MuPAD and is based on the inefficiency of the {\tt subs()} command.\\

\item If the term $X$ in an equation describing an infective class does not involve any state variables and is only a parameter, product of parameters, or quotient of parameters, $X \in \mathcal{F}$.  If terms like these exist, the disease-free subspace will not be invariant, and the model will not have a valid Next Generation $R_0$.

\item Every other term $X$ that does not satisfy the above will be placed in $\mathcal{V}$ for ODEs, or $\mathcal{T}$ for maps.

\end{enumerate}



\section{Template Examples}

\subsection{MAP Examples}

\subsubsection{Caswell 08}

This model is from Caswell's \textit{Perturbation analysis of nonlinear matrix population models}, 2008.  It involves two stages, juveniles ($x_1$) and adults ($x_2$).
%
\begin{equation*}
\xvec(t+1) = \begin{pmatrix} \sigma_1 (1-\gamma) & f \\ \sigma_1 \gamma & \sigma_2 \end{pmatrix} \xvec(t)
\end{equation*}
%
where the juvenile survival $\sigma_1(\xvec) = \tilde{\sigma} e^{-\textbf{e}^\textrm{T}\xvec}$, where $\textbf{e}$ is a vector of ones, $\sigma_2$ is the adult survival, $\gamma$ is the maturation probability, and $f$ is the adult fertility.  The parameter values given are $(f, \gamma, \tilde{\sigma}, \sigma_2) = ( 0.25, 1/15, 0.98, 0.95)$.

The main purpose of this model is to verify that {\sensai} gives the same equilibrium and sensitivities to those mentioned in the paper.  This model also serves as a great template for MAP examples.

\subsubsection{Hantavirus}

This model is from Allen and van den Driessche, \textit{The Basic Reproduction Number in Some Discrete-Time Epidemic Models}, 2008.  It involves susceptible and infected male and female rodents.
%
\begin{equation*}\label{eq:hantavirus}
\left.\begin{gathered}\begin{aligned}
S_m(t+1) &= \Big[ \frac{B}{2}+e^{-\beta_mI_m(t) - \beta_f I_f(t)}S_m(t)\Big]D(N)\\
I_m(t+1) &= \Big[ (1-e^{-\beta_mI_m(t)-\beta_f I_f(t)})S_m(t) + I_m(t)\Big]D(N)\\
S_f(t+1) &= \Big[ \frac{B}{2}+e^{-\beta_f I_m(t)-\beta_f I_f(t)}S_f(t)\Big] D(N)\\
I_f(t+1) &= \Big[ (1-e^{-\beta_f I_m(t)-\beta_f I_f(t)})S_f(t)+I_f(t)\Big]D(N)
\end{aligned}\end{gathered} \right \}
\end{equation*}
%
where the logistic growth is scaled by
%
\begin{equation*}
D(N) = \frac{K}{K+(b/2)N}.
\end{equation*}
%
where $K$ is the carrying capacity and $N$ is the total population; the harmonic mean birth function is
%
\begin{equation*}
B(N_m,N_f) = \frac{2bN_mN_f}{N}
\end{equation*}
%
where $N_m = S_m+I_m$ is the total number of males, $N_f = S_f+I_f$ is the total number of females, and $b>0$ is the average litter size, $k$ is the number of contacts that result in an infection, and $\beta_m$ and $\beta_f$ are the infection rate constants of males and females, respectively.  Parameter values are not provided in the paper, but some reasonable values are $K= 1000$, $\beta_f=0.09$, $\beta_m=0.9$, and $b=6$.

The main purpose of this model is to verify that {\sensai} gives the same value of $R_0$ as the paper.  This example exhibits two common practices in model formulation which {\sensai} performs extremely well \textit{in the Maple version}.  First, every term is scaled by the total population $N$.  Second, the infection rate is given by a probability of a non-infection not occurring.  Unfortunately, while {\sensai} handles the second issue, it is not able to handle the first.  This is due to MuPAD's ineffective {\tt subs()} command.

As an illustration, consider the following term.  Suppose there are two states $x_1$ and $x_2$ and the term in consideration is $p_1x_2 / (p_1 + x_1+x_2)$.  If $x_2$ is the infectious state, then this term belongs in $\mathcal{V}$, since the only appearance of a non-infective state is in the denominator, which is just a scaling by the total population.  This illustrates what occurs for the Hantavirus model with the scaling $D(N)$, with $p_1$ acting as $K$.  While the {\sf Maple} command {\tt subs(term, $x_1 + x_2 = N$)} converts this term to $p_1x_2 / (p_1 + N)$, the identical MuPAD command does not.  If $p_1$ were not in the denominator, the MuPAD command would work as desired.  Unfortunately, the computation of $R_0$ for models with the scaling of the total population is not available in the MuPAD version of {\sensai}, although it is available for the {\sf Maple} version.

\subsubsection{Six Stage Genetics}

This model is based on Caswell08, adding partially dominant selection based on viability and two alleles.

\subsection{ODE Examples}

\subsubsection{SIR}

This model is a typical SIR model with logistic growth.
%
\begin{equation*}\label{eq:SIR_log}
\left . \begin{gathered} \begin{aligned}
\frac{dS}{dt} &= rN \left( 1-\frac{N}{K}\right) - \beta SI - \delta S, \quad \\
\frac{dI}{dt} &= \beta SI - \gamma I - \mu I - \delta I, \\
\frac{dR}{dt} &= \gamma I - \delta R,
\end{aligned} \end{gathered} \right \}
\end{equation*}
%
where $N = S+I+R$ is the total population at any time $t$, $r$ is the per capita growth rate, $K$ is the carrying capacity, $\beta$ is the infection rate, $\delta$ is the natural death rate of the species, $\gamma$ is the recovery rate, and $\mu$ is the disease specific death rate.  Some reasonable parameter values are $r = 0.5$, $K=1000$, $\beta = 0.1$, $\delta = 0.2$, $\gamma$ = 0.02, and $\mu = 0.1$.

The main purpose of this model is that it is a standard ODE infection model with a simple $R_0$ that can be easily verified by hand.  This model also serves as a great template for ODE examples.

\subsubsection{SI (Indirect Transmission)}

This model is an SI model that involves indirect transmission of the infection.
%
\begin{equation*}\label{eq:background_si}
\left.\begin{gathered}\begin{aligned}
\frac{dS}{dt} &= rN\left(1-\frac{N}{K}\right)\\
\frac{dI}{dt} &= \beta -\gamma I
\end{aligned}\end{gathered}\right\}
\end{equation*}
%
where $N(t) = S(t) +I(t)$ is the total population, $r$ is the per capita growth rate, $K$ is the carrying capacity, $\beta$ is the background transmission probability, and $\gamma$ is the recovery rate.

The explaination of this model in detail can be found in \textit{Sensitivity Analysis of the Basic Reproduction Number and other Quantities for Infectious Disease Models}, Masters Thesis by Mikucki, 2012.  Parameter values may be chosen as  $r= 0.5$, $\beta = 0.8$, $\gamma = 0.02$, and $K = 1000$.

The main purpose of this model is to show that models with a background (indirect) transmission of the disease through the environment or some alternative source do not have a valid $R_0$.

\subsubsection{Plague}

This model is given by Buzby et. al. in \textit{Analysis of the sensitivity properties of a model of vector-borne bubonic plague}, 2008.  The first three classes are the SIR classes of rats, $N$ is the average number of fleas living on a rat, and $F$ is the number of free infectious fleas that are searching for a new host.

\begin{equation*}\left.\begin{gathered}\begin{aligned}
\dot{S}_R &= r_RS_R\left(1-\frac{T_R}{K_R} \right) + r_RR_R(1-p)-d_RS_R - \beta_R\frac{S_R}{T_R}F(1-e^{-aT_R}) \\
\dot{I}_R &= \beta_R\frac{S_R}{T_R}F(1-e^{-aT_R})-(d_R+m_R)I_R\\
\dot{R}_R &= r_RR_R\left(p-\frac{T_R}{K_R}\right) + m_Rg_RI_R -d_RR_R\\
\dot{N} &= r_FN\left(1-\frac{N}{K_F}\right) + \frac{d_F}{T_R}F(1-e^{-aT_R})\\
\dot{F} &= (d_R+m_R(1-g_R))I_RN-d_FF
\end{aligned}\end{gathered}\right\}\end{equation*}
%
where $T_R = S_R+I_R+R_R$ is the total size of the rat population, $r_R$ is the net rat reproduction rate, $K_R$ is the rat carrying capacity, $p$ is the proportion of offspring that inherity the disease, $d_R$ is the natural rat death rate, $\beta_R$ is the transmission rate from rats to fleas, $m_R$ is the rate that rats leave the infected class, $g_R$ is the fraction of rates that become resistant, $a$ is the searching efficiency of the fleas, $r_F$ is the net flea reproductive rate, $d_F$ is the natural flea death rate, and $K_F$ is the flea carrying capacity.

\subsubsection{Dengue}

This model is given by Garba in \textit{Backward bifurcations in dengue transmission dynamics}, 2008.  It is an SEIR model that descirbes the dynamics of dengue fever, an infection carried by a vector, mosquiteos.  The equations are
%
\begin{equation*}\label{eq:dengue}
\left. \begin{gathered} \begin{aligned}
\frac{dS_H}{dt} &= \Pi_H-\lambda_HS_H-\mu_HS_H\\
\frac{dE_H}{dt} &= \lambda_HS_H-(\sigma_H+\mu_H)E_H\\
\frac{dI_H}{dt} &= \sigma_HE_H-(\tau_H+\mu_H+\delta_H)I_H\\
\frac{dR_H}{dt} &= \tau_HI_H-\mu_HR_H\\
\frac{dS_V}{dt} &= \Pi_V-\lambda_VS_V-\mu_VS_V\\
\frac{dE_V}{dt} &= \lambda_VS_V-(\sigma_V+\mu_V)E_V\\
\frac{dI_V}{dt} &= \sigma_VE_V-(\mu_V+\delta_V)I_V
\end{aligned} \end{gathered}\right \}
\end{equation*}
%
where $\lambda_H = \frac{C_{HV}}{N_H}(\eta_VE_V+I_V)$ is the human infection rate, $\lambda_V = \frac{C_{HV}}{N_H}(\eta_HE_H+I_H)$ is the vector infection rate, and $N_H = S_H+E_H+I_H+R_H$ is the total human population.  The parameter values provided are $\mu_H=0.0195$, $\sigma_H=0.5300$, $\Pi_H=10$, $\delta_H=0.9900$, $\eta_H=0.9900$, $\tau_H=0.2000$, $\mu_V=0.0140$, $\sigma_V = 0.2000$, $\Pi_V = 30$, $\delta_V = 0.0057$, $\eta_V=0.9800$, and
$C_{HV} = 0.038$.

An interesting initial condition for the model is $\textbf{x}_\textbf{0} = (\frac{\Pi_H}{\mu_H}, 0, 0, 0, \frac{\Pi_V}{\mu_V}, 0, 200)$, which will show that even thought $R_0 <1$, infection may still persist in the population.

\subsubsection{Typhoid}

This model is given by Bailey and Duppenthaler in \textit{Sensitivity Analysis in the Modelling of Infectious Disease Dynamics}, 1980.  It is a 9-stage SIR type model where $x_1$ = susceptibles, $x_2$ = incubating noninfectious, $x_3$ = incubating infectious, $x_4$ = sick infectious, $x_5$= sick noninfectious, $x_6$ =  temporary carrier, $x_7$ = permanent carrier, $x_8$ = short resistance, and $x_9$ = long resistance.
%
\begin{equation*}\label{eq:typhoid}
\left.\begin{gathered}\begin{aligned}
\dot{x_1} &= -(\rho_{12}+\rho_{13})x_1y + \rho_{41}x_4+\rho_{51}x_5+\rho_{61}x_6+\rho_{81}x_8+\rho_{91}x_9-\mu x_1+\mu\\
\dot{x_2} &= \rho_{12}x_1y - (\rho_{23}+\rho_{24} +\rho_{25}+\mu)x_2 + \rho_{32}x_3\\
\dot{x_3} &= \rho_{13}x_1y - (\rho_{32}+\rho_{34}+\rho_{35}+\mu)x_3 + \rho_{23}x_2\\
\dot{x_4} &= \rho_{24}x_2 +\rho_{34}x_3 + \rho_{54}x_5-(\rho_{41}+\rho_{45}+\rho_{46}+\rho_{48}+\mu)x_4\\
\dot{x_5} &= \rho_{25}x_2 + \rho_{35}x_3+\rho_{45}x_4-(\rho_{51}+\rho_{54}+\rho_{58}+\mu)x_5 \\
\dot{x_6} &= \rho_{46}x_4-(\rho_{61}+\rho_{67}+\rho_{68}+\mu)x_6\\
\dot{x_7} &= \rho_{67}x_6 - \mu x_7\\
\dot{x_8} &= \rho_{48}x_4 + \rho_{58}x_5 + \rho_{68}x_6 - (\rho_{81}+\rho_{89}+\mu)x_8\\
\dot{x_9} &= \rho_{89}x_8 - (\rho_{91}+\mu)x_9
\end{aligned}\end{gathered}\right \}
\end{equation*}

Parameter values are provided by Bailey are as follows:  $\rho_{12} = 8.43381\times10^{-3}$, $\rho_{13} = 8.51900\times10^{-5}$,  $\rho_{23} = 2.85720\times10^{-3}$, $\rho_{24} = 6.78585\times10^{-2}$, $\rho_{25} = 7.14300\times10^{-4}$, $\rho_{32} = 7.14300\times10^{-4}$, $\rho_{34} = 6.42870\times10^{-2}$, $\rho_{35} = 6.42870\times10^{-3}$, $\rho_{41} = 3.46000\times10^{-3}$, $\rho_{45} = 3.46000\times10^{-3}$, $\rho_{46} = 3.46000\times10^{-3}$, $\rho_{48} = 2.40124\times10^{-2}$, $\rho_{51} = 3.46000\times10^{-3}$, $\rho_{54} = 6.92000\times10^{-3}$, $\rho_{58} = 2.40124\times10^{-2}$, $\rho_{61} = 1.11100\times10^{-3}$, $\rho_{67} = 3.33300\times10^{-3}$, $\rho_{68} = 6.66600\times10^{-3}$, $\rho_{81} = 2.74000\times10^{-4}$, $\rho_{89} = 2.46600\times10^{-3}$, $\rho_{91} = 2.74000\times10^{-4}$, and $\mu = 5.48000\times10^{-5}$.

This example demonstates {\sensai}'s ability to implement a large system and compute $R_0$ effectively.  In this model, equations 2-7 are considered infective, so the next generation matrix is a $6\times 6$ matrix with analytical (not numerical) components.

\end{document}
